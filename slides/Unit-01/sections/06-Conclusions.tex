\section{Conclusions}
\subsection{}

\begin{frame}
\frametitleTC{Conclusions}
\framesubtitleTC{Recap and lessons learnt (1/3)}
\myPause
 \begin{itemize}[<+-| alert@+>]
 \item Control is \TC{governing a phenomenon} (physical or not).
 \item This means determining \TC{actions} on that phenomenon, based on \TC{requirements}\\
       and possibly \TC{measurements}.
 \item Major issues are exogenous actions -- or \TC{disturbances} -- and \TC{uncertainty},\\
       i.e., only partial knowledge of the controlled system.
 \item Quite often a control problem can be formulated in terms of signals,\\
       i.e., as \TC{tracking a reference} and/or \TC{rejecting disturbances}.
 \end{itemize}
\end{frame}

\begin{frame}
\frametitleTC{Conclusions}
\framesubtitleTC{Recap and lessons learnt (2/3)}
\myPause
 \begin{itemize}[<+-| alert@+>]
 \item A controller may or may not know the effects of its action on the controlled system, giving rise to
       \TC{closed-loop} (or \TC{feedback}) versus \TC{open-loop} control.
 \item A controller can operate in the continuous time, but in computing systems this is\\
       practically never the case;
 \item hence for us a controller is a piece of code, invoked either periodically or based\\
       on events, giving rise to \TC{discrete-time} versus \TC{event-based} control.
 \item Control signals can be \TC{numeric} or \TC{lexical}, corresponding to \TC{modulating}\\
       versus \TC{logic} control.
 \item Control problems are posed and treated with reference to one\\
       or more classes of \TC{dynamic systems}. 
 \end{itemize}
\end{frame}

\begin{frame}
\frametitleTC{Conclusions}
\framesubtitleTC{Recap and lessons learnt (3/3)}
\myPause
 \begin{itemize}[<+-| alert@+>]
 \item Dynamic systems have a \TC{state}, hence they
       \begin{itemize}[<+-| alert@+>]
       \item remember the past,
       \item or -- equivalently -- their initial condition,
       \item and thus \TC{can produce different outputs in response to the same input}...
       \item ...without however being time-varying or ``adaptive'' (remember this remark!).
       \end{itemize}
 \item We have seen some classes of dynamic systems.
 \item In the following we shall concentrate on LTI ones, almost exclusively\\
       in the DT domain. 
 \end{itemize}
\end{frame}

\section{}
{
\setbeamertemplate{headline}{
  \begin{beamercolorbox}[wd=\paperwidth,ht=4.2ex,dp=1.5ex]{palette quaternary}
  \end{beamercolorbox}
  }
\setbeamertemplate{footline}{
  \begin{beamercolorbox}[wd=\paperwidth,ht=2.2ex,dp=1.5ex]{palette quaternary}
  \end{beamercolorbox}
  }
\begin{frame}[noframenumbering]
 \vspace{20mm}\Huge{Discussion open}
\end{frame}
}

