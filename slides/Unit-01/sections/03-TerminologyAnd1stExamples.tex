\section{Terminology and first examples}
\subsection{}

\begin{frame}
\frametitleTC{Terminology}
\framesubtitleTC{(1/2)}
\myPause
 \begin{itemize}[<+-| alert@+>]
 \item \TC{System} -- when required to avoid ambiguity, \TC{\underline{controlled} system:}
       \begin{itemize}[<+-| alert@+>]
       \item[] the object or phenomenon to be governed
       \item[] (a vehicle, the thermal behaviour of a building,...).
       \end{itemize}
 \item \TC{Requirements} or \TC{objectives:}
       \begin{itemize}[<+-| alert@+>]
       \item[] what you want the system to do\\
       \item[] (cruise as close as possible to 50 km/h, keep temperature between 18$^{\circ}$C\\
                and 21$^{\circ}$C in all rooms,...)
       \end{itemize}
 \item \TC{Controls} or \TC{control actions} or simply \TC{actions:}
       \begin{itemize}[<+-| alert@+>]
       \item[] what is done to the system with the purpose of fulfilling the requirements
       \item[] (act on the gas/brakes, modulate fuel flowrate to heater,\\
                operate fan coils,...).
       \end{itemize}
 \item \TC{Outcomes:}
       \begin{itemize}[<+-| alert@+>]
       \item[] the actual behaviour of the system
       \item[] (real speed and temperature behaviour over time,...).
       \end{itemize}
 \end{itemize}
\end{frame}

\begin{frame}
\frametitleTC{Terminology}
\framesubtitleTC{(2/2)}
\myPause
 \begin{itemize}[<+-| alert@+>]
 \item \TC{Disturbances:}
       \begin{itemize}[<+-| alert@+>]
       \item[] anything that affects the outcomes and cannot be manipulated, but possibly sensed
       \item[] (road slope, wind, vehicle load, weather, room occupancy,...);
       \item[] in other words, actions from the environment to which the system must be resilient.
       \end{itemize}
 \item \TC{Sensors:}
       \begin{itemize}[<+-| alert@+>]
       \item[] the entities that gather information from the system
       \item[] (tacho/accelerometer, GPS, inside/outside temperature sensors,...).
       \end{itemize}
 \item \TC{Actuators:}
       \begin{itemize}[<+-| alert@+>]
       \item[] the entities that act on the system to exert the actions
       \item[] (engine, brakes, burners, pumps, valves,...).
       \end{itemize}
 \item \TC{Controllers:}
       \begin{itemize}[<+-| alert@+>]
       \item[] the entities that determine the actions given the information\\
               deemed relevant\\
       \item[] (the object of your design).
       \end{itemize}
 \end{itemize}
\end{frame}

\begin{frame}
\frametitleTC{Remarks}
\framesubtitleTC{}
\myPause
 \begin{itemize}[<+-| alert@+>]
 \item Quite often we can talk about a control \TC{objective} in terms of a \TC{signal}\\
       following another one.
 \item In the case we use to say that
       \begin{itemize}[<+-| alert@+>]
       \item we have a \TC{controlled variable} (e.g., the speed of a vehicle)
       \item and want it to follow a \TC{set point} or \TC{reference} signal (e.g., go from 0 to 100 km/h
             linearly in 10 seconds).
       \end{itemize}
 \item Also, quite often our action can be viewed as setting a variable (e.g., the gas\\
       valve opening in the 0--100\% range); we call such a variable\\
       the \TC{control signal}.
 \item Consistently, we can often represent exogenous actions (e.g., wind\\
       speed and directions) that we called \TC{disturbances}, as signals.
 \end{itemize}
\end{frame}

\begin{frame}
\frametitleTC{Remarks}
\framesubtitleTC{}
\myPause
 \begin{itemize}[<+-| alert@+>]
 \item You are probably thinking that in the computer world the situation just sketched is hardly ever encountered.
 \item The objection seems in fact reasonable, as the entities one manages in computers are far more
       heterogeneous than mere numbers varying over time.
 \item \vfill Nonetheless hold on; you will discover that on the contrary, much more computer-related problems
       than one expects, can be formulated in terms\\ of \TC{set point following} and/or \TC{disturbance rejection}.
 \item In the end, learning systems and control is learning a new kind\\
       of abstraction.
 \end{itemize}
\end{frame}


