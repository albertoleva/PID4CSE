\section{Foreword and prerequisites}
\subsection{}

\begin{frame}
\frametitleTC{What we are going to do together}
\framesubtitleTC{in a nutshell}
\myPause
\begin{quote}
If you search the literature for ``control in computers'', you will find first that there is a lot of material, and then that the so-called ``PID'' controller is one of the most widely employed.\\
However, as we shall see, an incorrect use of that object can be as disastrous as a good one beneficial. Hence you need to understand and master the underlying principles, that is, the essentials of the systems and feedback control theory.\\
This activity will make you capable in the first place of using PID control knowledgeably, and as a by-product, will also teach you what a dynamic model of a system is.\\
In turn, this is useful for many purposes, including e.g. to detect\\
when a problem cannot be tackled with a PID, and thus you need\\
to learn more control, or ask for advice from a control specialist,\\
or both.
\end{quote}
\end{frame}

\begin{frame}
\frametitleTC{Prerequisites}
\framesubtitleTC{(1/3) mathematics}
\myPause
\begin{itemize}[<+-| alert@+>]
\item Basic knowledge of differential/integral calculus:
      \begin{itemize}[<.(1)->]
      \item limit (finite and infinite),
      \item derivative and integral in one variable.
      \end{itemize}
      \myPause
\item Complex numbers and operations on them.
\item Basic knowledge of matrix algebra:
      \begin{itemize}[<.(1)->]
      \item matrix operations,
      \item determinant, inverse,
      \item eigenvalues and eigenvectors.
      \end{itemize}
\end{itemize}
\end{frame}


\begin{frame}
\frametitleTC{Prerequisites}
\framesubtitleTC{(2/3) physics --- yes we do need a little bit for some examples}
\myPause
\begin{itemize}[<+-| alert@+>]
\item Mass balance: the time derivative of the mass $M$ contained in a volume\\
      (e.g., a tank) is the sum of the $n$ signed mass flowrates $w_i$ at the $n$ boundaries\\
      (e.g., pipes) of that volume, that is,
      \begin{displaymath}
       \frac{dM(t)}{dt} = \sum_{i=1}^n w_i(t).
      \end{displaymath}
\item Energy balance: the time derivative of the energy $E$ contained in a volume\\
      (e.g., a solid body) is the sum of the $n$ signed powers $q_i$ at the $n$\\
      boundaries (e.g., surfaces) of that volume, that is,
      \begin{displaymath}
       \frac{dE(t)}{dt} = \sum_{i=1}^n q_i(t).
      \end{displaymath}
\item We evidently take the convention ``positive means entering''.
\end{itemize}
\end{frame}

\begin{frame}
\frametitleTC{Prerequisites}
\framesubtitleTC{(3/3) physics --- yes we do need a little bit for some examples}
\myPause
\begin{itemize}[<+-| alert@+>]
\item Convective/conductive heat transfer: the power $q_{ab}$ transferred form a body at temperature $T_a$
      to another at temperature $T_b$ is proportional by a \emph{thermal conductance} $G_{ab}$ to the temperature
      difference, that is,
      \begin{displaymath}
       q_{ab}(t) = G_{ab} \left( T_a(t)-T_b(t) \right).
      \end{displaymath}
\item Anticipation: later on we shall see more balances of the \emph{dynamic} type, i.e.,\\
      taking the form``the time derivative of something equals the sum\\
      of something else''.
\item You may be surprised how frequent these are in computing systems,\\
      hence how many different problems can be treated uniformly.
\end{itemize}
\end{frame}

