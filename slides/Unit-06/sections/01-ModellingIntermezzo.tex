\section{A modelling intermezzo}
\subsection{}

\begin{frame}
\frametitleTC{Foreword and motivation}
\framesubtitleTC{}
\myPause
 \begin{itemize}[<+-| alert@+>]
 \item We said right at the outset, that this activity would focus on control\\
       and not on modelling.
 \item However, without some knowledge on that matter, interpreting the\\
       parameters of a control may be extremely cumbersome.
 \item Hence we go through the bare minimum we need.
 \item In particular, we temporarily bring the continuous-time (CT) domain\\
       back into play.
 \item \vfill CAVEAT: this material is not presented ``the mainstream way'',\\
       so please be EXTREMELY attentive, stop and ask questions\\
       immediately at the minimum unclear statement.
 \end{itemize}
\end{frame}


\begin{frame}
\frametitleTC{Foreword and motivation}
\framesubtitleTC{}
\myPause
 \begin{itemize}[<+-| alert@+>]
 \item Any phenomenon naturally takes place in the CT domain.
 \item However we (shall) realise controllers as algorithms to run when a new value for $u$ is needed.
 \item AS A RESULT OF THIS we need to model the controlled object as something that evolves ``by steps''.
 \item We have three important questions to answer.
       \begin{enumerate}[<+-| alert@+>]
       \item Does the physics of that object ``naturally'' tell us when a new $u$ is to be\\
             computed, or not (i.e., \underline{we} have to decide when to compute)?
       \item In the second case, assuming that we are computing $u$ periodically,\\
             is there any clue to choose the period?
       \item Still in the second case, and most important to parametrise\\
             a controller meaningfully, does the process model $P(z)$ change\\
             if we change the period?
       \end{enumerate}
 \item \vfill As usual, we start from some examples and then abstract.
 \end{itemize}
\end{frame}

\begin{frame}
\frametitleTC{Example 1}
\framesubtitleTC{Preemptive scheduling}
\myPause
 \begin{itemize}[<+-| alert@+>]
 \item We look at the controlled system (plus sensor and actuator), NOT at the control.
 \item Let us focus on the \emph{control oriented} model for one task.
 \item Functionally:
       \begin{itemize}[<+-| alert@+>]
       \item a CPU time amount (or \TC{burst}) $b$ is allotted by some control (not our business here);
       \item a timer is set and the task activated (actuation);
       \item there is \TC{no information} from the task (no assumption on its code can\\
             be made, the OS has to be agnostic) till either the time elapses,\\
             or the task yields the CPU;
       \item the scheduler regains control and measures (sensing) the actually\\
             used CPU time; this will in general be $b+\delta b$, where $\delta b$\\
             is a disturbance (yield before $b$, preemption interrupt received\\
             while in a critical section, and so on);
       \item control is invoked, and the same or some other task is activated.
       \end{itemize}
 \end{itemize}
\end{frame}

\begin{frame}
\frametitleTC{Example 1}
\framesubtitleTC{Preemptive scheduling}
\myPause
 \begin{itemize}[<+-| alert@+>]
 \item Is there a ``natural cadence'' for computing a new control? 
 \item[] $\Rightarrow$ Yes, that of task activations.
 \item Having thus $k$ count the scheduler intervention, which is the model with $b(k)$ as input 
       and the task accumulated CPU time $t_{CPU}(k)$ as output?
 \item[] $\Rightarrow$ Quite immediately, assuming that instant $k$ is at the \emph{end} of the activation\\
       \hspace{5.5mm}period, thus the relative burst was decided at $k-1$,
       \begin{displaymath}
        t_{CPU}(k) = t_{CPU}(k-1)+b(k-1)+\delta_b(k-1).
       \end{displaymath}
 \item Does the model depend on the (continuous) time elapsed between\\
       interventions $k-1$ and $k$?
 \item[] $\Rightarrow$ Apparently, no.    
 \end{itemize}
\end{frame}

\begin{frame}[fragile]
\frametitleTC{Example 2}
\framesubtitleTC{CPU thermal management --- brutally simplified}
\myPause
 \begin{itemize}[<+-| alert@+>]
 \item Once again, no control and just the phenomenon to govern.
 \item Thermal generation/storage/dissipation takes place in the CT domain:
       \begin{itemize}[<+-| alert@+>]
       \item {\scriptsize\verb#stored energy             = CPU thermal capacity * CPU temperature#}
       \item {\scriptsize\verb#time derivative of energy = generated power - dissipated power#}
       \item {\scriptsize\verb#generated power           = input#}
       \item {\scriptsize\verb#dissipated power          = sink thermal conductance * (CPU temp. - ambient temp.)#}
       \item {\scriptsize\verb#ambient temperature       = another input#}
       \end{itemize}
 \item As a differential equation, then,
       \begin{displaymath}
        C_{CPU} \frac{dT_{CPU}(t)}{dt} = P(t) - G_{sink}(T_{CPU}(t)-T_{amb}(t)).
       \end{displaymath}
 \end{itemize}
\end{frame}

\begin{frame}
\frametitleTC{Example 2}
\framesubtitleTC{CPU thermal management}
\myPause
 \begin{itemize}[<+-| alert@+>]
 \item We want a DT model, however, so we decide a timestep $T_s$ to compute the model state and output (here
       just $T_{CPU}$) and \TC{replace the time derivative with the incremental ratio over one step}.
 \item Writing $v(k)$ in the DT to indicate $v(kT_s)$ in the CT, whatever $v$ is, this gives
       \begin{displaymath}
        C_{CPU} \frac{T_{CPU}(k)-T_{CPU}(k-1)}{T_s} = P(k) - G_{sink}(T_{CPU}(k)-T_{amb}(k)).
       \end{displaymath}
 \item The curious may ask why $k$ and not $k-1$ on the right hand side.\\
       We omit the matter in this course, but if interested look for\\
       ``implicit/explicit discretisation''.
 \end{itemize}
\end{frame}

\begin{frame}[fragile]
\frametitleTC{Example 2}
\framesubtitleTC{CPU thermal management}
\myPause
 \begin{itemize}[<+-| alert@+>]
 \item Now getting the model in $z$ form is straightforward:
       \begin{displaymath}
        C_{CPU} \frac{T_{CPU}-z^{-1}T_{CPU}}{T_s} = P - G_{sink}(T_{CPU}-T_{amb})
       \end{displaymath}
 \item Maxima:
       {\scriptsize
       \begin{verbatim}
  dum: solve(Ccpu*(Tcpu-Tcpu/z)/Ts=P-Gsink*(Tcpu-Tamb),Tcpu); // Solve for Tcpu
  sol: rhs(dum[1]);                                           // Take RHS of 1st element
  TFs: jacobian([sol],[P,Tamb]);                              // Transfer functions
       \end{verbatim}
       }
 \item Result:
       \begin{displaymath}
        \begin{array}{rcl}
         T_{CPU}(k) &=&  \frac{zT_s}{(C_{CPU}+G_{sink}Ts)z-C_{CPU}} \, P(k) \\ \\
                    & & +\frac{zG_{sink}T_s}{(C_{CPU}+G_{sink}Ts)z-C_{CPU}} \, T_{amb}(k).
        \end{array}
       \end{displaymath}
 \end{itemize}
\end{frame}

\begin{frame}
\frametitleTC{Example 2}
\framesubtitleTC{CPU thermal management}
\myPause
 \begin{itemize}[<+-| alert@+>]
 \item Does the model depend on the continuous time elapsed between $k-1$ and $k$,\\
       i.e., on $T_s$?
 \item[] $\Rightarrow$ Apparently, yes: $T_s$ appears as a parameter in the transfer functions.
 \item Is there a ``natural cadence'' for computing a new control for this model?
 \item Reformulating, is there a ``good'' value of $T_s$ so that the points computed by the\\
       DT model represent ``well enough'' the CT solution, so as to be informative\\
       for control and thereby suggest the cadence above?
 \item[] $\Rightarrow$ Yes, but that ``good'' $T_s$ apparently depends on the numbers\\
       \hspace{5.5mm}in the model. We have to decide based on them.
 \end{itemize}
\end{frame}

\begin{frame}
\frametitleTC{Lessons learnt}
\framesubtitleTC{}
\myPause
 \begin{itemize}[<+-| alert@+>]
 \item Sometimes $k$ just counts control interventions, and the time in between them does not change the process
       model $P(z)$.
 \item In this case the control system is DT, and one can reason entirely in the DT domain. No need to relate
       control parameters to any CT entity.
 \item \vspace{3mm}Sometimes, conversely, the model $P(z)$ seen by the controller depends on the time between
       two evaluations of its output, which for us coincides with the cadence to compute the control signal.
 \item In this case the control system is DT but also \TC{sampled-signals},\\ 
       and for evident practical reasons, control parameters need\\
       expressing in such a way to not change if $T_s$ is changed.
 \item Of course, in this case clues to select $T_s$ are also needed.
 \end{itemize}
\end{frame}

