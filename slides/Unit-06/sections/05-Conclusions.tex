\section{Conclusions}
\subsection{}

\begin{frame}
\frametitleTC{Conclusions}
\framesubtitleTC{Recap and lessons learnt (1/2)}
\myPause
 \begin{itemize}[<+-| alert@+>]
 \item Distinguishing \TC{pure DT} and \TC{sampled-signals} cases.
 \item Interpreting a CT model in the DT domain by replacing the time derivative\\
       with the one-step incremental ratio, and \emph{vice versa}.
 \item Understanding that control needs SIMPLE models, that these can come from both\\
       knowledge of the phenomenon and data, and that in general complexity is NOT a merit if not justified.
 \item Being cautious -- not necessarily negative, but cautious -- in the face of ``benchmark sets'' not qualified in a system-theoretical manner. 
 \item Tuning a PI controller for a dominantly first-order asymptotically\\
       stable process.
 \item Dealing with a (purely) integrating process.
 \end{itemize}
\end{frame}

\begin{frame}
\frametitleTC{Conclusions}
\framesubtitleTC{Recap and lessons learnt (2/2)}
\myPause
 \begin{itemize}[<+-| alert@+>]
 \item Understanding the role of the P, I, and D actions (prompt reaction, zero\
       steady-state error, error anticipation).
 \item Understanding \TC{(anti)windup} and \TC{tracking}.
 \item Writing a PI algorithm as procedural code.
 \item Writing a PID algorithm and discussing arbitrary choices as for state management\\
      (one constraint on $u$, two past values to determine for $u_I$ and $u_D$).
 \item \vfill Next comes a practice session.
 \end{itemize}
\end{frame}

\section{}
{
\setbeamertemplate{headline}{
  \begin{beamercolorbox}[wd=\paperwidth,ht=4.2ex,dp=1.5ex]{palette quaternary}
  \end{beamercolorbox}
  }
\setbeamertemplate{footline}{
  \begin{beamercolorbox}[wd=\paperwidth,ht=2.2ex,dp=1.5ex]{palette quaternary}
  \end{beamercolorbox}
  }
\begin{frame}[noframenumbering]
 \vspace{20mm}\Huge{Discussion open}
\end{frame}
}

