\section{Exercise 03}
\subsection{}

\begin{frame}
\frametitleTC{Problem}
\framesubtitleTC{This one we solve together}
 Given the two DT LTI dynamic systems
 \begin{displaymath}
  S_1:
  \left\{\begin{array}{rcl}
   x_1(k) &=& 0.5x_1(k-1)+u(k-1)\\
   x_2(k) &=& 0.5x_2(k-1)+u(k-1)\\
   y(k)   &=& x_1(k)+x_2(k)
  \end{array}\right. \quad
  S_2:
  \left\{\begin{array}{rcl}
   x_1(k) &=& 0.4x_1(k-1)+u(k-1)\\
   x_2(k) &=& 0.8x_2(k-1)+u(k-1)\\
   y(k)   &=& x_2(k)
  \end{array}\right.
 \end{displaymath}

 \begin{itemize}[<+-| alert@+>]
 \item[(a)] compute their transfer functions,
 \item[(b)] comment on the result.
 \end{itemize}
\end{frame}

\begin{frame}
\frametitleTC{Solution}
\framesubtitleTC{Item (a)}
\myPause
 \begin{itemize}[<+-| alert@+>]
 \item We start with $S_1$:
       \begin{itemize}
       \item[] \begin{itemize}[<+-| alert@+>]
               \item[$G(z)$] \vspace{1mm}
                    $= \begin{bmatrix} 1 & 1  \end{bmatrix} \,
                       \begin{bmatrix} z-0.5 & 0 \\
                                       0     & z-0.5 \end{bmatrix}^{-1} \,
                       \begin{bmatrix} 1 \\ 1 \end{bmatrix}
                       +0
                    $
               \item[] \vspace{1mm}
                    $= \cfrac{1}{(z-0.5)^2}
                       \begin{bmatrix} 1 & 1 \end{bmatrix} \,
                       \begin{bmatrix} z-0.5 & 0 \\
                                       0     & z-0.5 \end{bmatrix} \,
                       \begin{bmatrix} 1 \\ 1 \end{bmatrix}
                    $
               \item[] \vspace{1mm}
                    $= \cfrac{1}{(z-0.5)^2}
                       \begin{bmatrix} z-0.5 & z-0.5 \end{bmatrix} \,
                       \begin{bmatrix} 1 \\ 1 \end{bmatrix}
                    $
               \item[] \vspace{1mm}
                    $= \cfrac{2\cancel{(z-0.5)}}{(z-0.5)^{\cancel{2}}} \qquad \qquad
                       \TC{\leftarrow \text{zero/pole \underline{cancellation}}}
                    $
               \item[] \vspace{1mm}
                    $= \cfrac{2}{z-0.5}.
                    $
               \end{itemize}
       \end{itemize}
 \end{itemize}
\end{frame}


\begin{frame}[fragile]
\frametitleTC{Solution}
\framesubtitleTC{Item (a)}
\myPause
 \begin{itemize}[<+-| alert@+>]
 \item And now $S_2$:
       \begin{itemize}
       \item[] \begin{itemize}[<+-| alert@+>]
               \item[] wxMaxima, we are lazy \smiley
                       \begin{verbatim}
A:matrix([0.4,0],[0,0.8]);
b:matrix([1],[1]);
c:matrix([0,1]);
G:c.invert(z*ident(2)-A).b;
                       \end{verbatim}
               \item[] \vspace{1mm}
                    $ G(z) = \cfrac{1}{z-0.8}.
                    $
               \end{itemize}
               \item \vspace{3mm}Another case with cancellation: the characteristic polynomial of $A$\\
                     has degree 2, the denominator of $G(z)$ has degree 1.
       \end{itemize}
 \end{itemize}
\end{frame}



\begin{frame}
\frametitleTC{Solution}
\framesubtitleTC{Item (b) -- comments on the $S_1$ case}
\myPause
 \begin{itemize}[<+-| alert@+>]
 \item In fact $S_1$ is a ``fake'' 2nd order system, since as far as induced motions are\\
       concerned  $x_1$ and $x_2$ are the same, and the system is \TC{input-output} (i.e, as far\\
       as we only look at $u$ and $y$) \TC{equivalent} to a 1st order one:
       \begin{displaymath}
        \left\{\begin{array}{rcl}
         x_1(k) &=& 0.5x_1(k-1)+u(k-1)\\
         x_2(k) &=& 0.5x_2(k-1)+u(k-1)\\
         y(k)   &=& x_1(k)+x_2(k)
        \end{array}\right. \quad
        \Rightarrow \quad
        \left\{\begin{array}{rcl}
         x(k) &=& 0.5x(k-1)+u(k-1)\\
         y(k) &=& 2x(k)
        \end{array}\right.
        \end{displaymath} 
        apparently with
        \begin{displaymath}
         G(z)=\cfrac{2}{z-0.5}.
        \end{displaymath} 
 \item This system cannot move in the whole $(x_1,x_2)$ plane, but only\\
       on the straight line $x_1=x_2$.
 \item In general, systems like this can only move on a \TC{subspace} (line)\\
       of their \TC{state space} (plane).
 \item We call them \TC{not fully reachable} (the term should be intuitive).
 \end{itemize}
\end{frame}

\begin{frame}
\frametitleTC{Solution}
\framesubtitleTC{Item (b) -- comments on the $S_2$ case}
\myPause
 \begin{itemize}[<+-| alert@+>]
 \item Also $S_2$ is a ``fake'' 2nd order system, because however $x_1$ moves, $y$ does not
       reveal this, and thus we can reduce this system as well to a 1st order one:
       \begin{displaymath}
        \left\{\begin{array}{rcl}
         x_1(k) &=& 0.4x_1(k-1)+u(k-1)\\
         x_2(k) &=& 0.8x_2(k-1)+u(k-1)\\
         y(k)   &=& x_2(k)
        \end{array}\right. \quad
        \Rightarrow \quad
        \left\{\begin{array}{rcl}
         x(k) &=& 0.8x(k-1)+u(k-1)\\
         y(k) &=& x(k)
        \end{array}\right.
        \end{displaymath} 
        apparently with
        \begin{displaymath}
         G(z)=\cfrac{1}{z-0.8}.
        \end{displaymath} 
 \item In this system one state variable does not influence the output.
 \item We call such systems \TC{not fully observable} (the term should be\\
       once again intuitive).
 \end{itemize}
\end{frame}

\begin{frame}
\frametitleTC{Proposed exercise 03}
\framesubtitleTC{Try this at home, ask questions next time if needed}
\myPause
 Take the two system of exercise 03, and turn them into block diagrams.\\
 \vspace{5mm}Analyse those diagrams along the idea that parts of the system are not influenced
 by the input, or do not influence the output. 
\end{frame}



\begin{frame}
\frametitleTC{Takeaways}
\framesubtitleTC{from exercise 03 (and the proposed one)}
\myPause
 \begin{itemize}[<+-| alert@+>]
 \item There are cases in which a system cannot move in all its state space, or equivalently,
       some state variables are not influenced by the input.
 \item Suggestion for reflections: why ``equivalently''?
 \item There are also cases in which some state variables do not influence the output.
 \item There are also cases where both facts occur.
 \item We detect this because in computing the transfer function we get\\
       \TC{cancellations}.
 \end{itemize}
\end{frame}

\begin{frame}
\frametitleTC{Takeaways}
\framesubtitleTC{from exercise 03 (and the proposed one)}
\myPause
 \begin{itemize}[<+-| alert@+>]
 \item Do we need to bother?
 \item Yes, because this collectively means that a system may have \TC{hidden parts}\\
       (things the transfer function does not tell).
 \item In setting up controls, we must be careful to not generate UNSTABLE hidden parts.
 \item We do not mind about asymptotically stable ones, because their effect\\
       vanishes with their free motion.
 \end{itemize}
\end{frame}



