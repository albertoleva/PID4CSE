\section{Wrap-up}
\subsection{}

\begin{frame}
\frametitleTC{Ideas and capabilities to take home}
\framesubtitleTC{(1/2) not necessarily in the same order as we saw them...}
 \begin{itemize}[<+-| alert@+>]
 \item Interpreting the state space description of a DT LTI dynamic system\\
       as scalar equations.
 \item Computing a system's transfer function from its state space description.
 \item Computing state and output responses in the time domain, and distinguishing\\
       free and induced motion.
 \item Commonly used signals: impulse, step, ramp.
 \end{itemize}
\end{frame}

\begin{frame}
\frametitleTC{Ideas and capabilities to take home}
\framesubtitleTC{(2/2)}
 \begin{itemize}[<+-| alert@+>]
 \item Poles and zeroes of a transfer functions, the former being eigenvalues\\
       of the dynamic matrix.
 \item Pole/zero cancellations and hidden parts.
 \item Writing a dynamic system as block diagram and recognise the inherent\\
       feedback in it.
 \item Combining blocks into overall transfer functions: for the moment series and parallel, more
       on this subject later on.
 \item A possible way to transform a transfer function into a state space\\
       representation.
 \item Stability and eigenvalues of the dynamic matrix (for the curious,\\
       we proved our statement by writing that matrix in \TC{Jordan} form).
 \end{itemize}
\end{frame}

