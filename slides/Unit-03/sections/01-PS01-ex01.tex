\section{Exercise 01}
\subsection{}

\begin{frame}
\frametitleTC{Problem}
\framesubtitleTC{This one we solve together}
\myPause
 Given the DT LTI dynamic system described in the state space by
 \begin{displaymath}
  A = \begin{bmatrix} 0.4 & 0.4 \\ 0.05 & 0.3 \end{bmatrix}, \quad
  b = \begin{bmatrix} 1 \\ 0.5 \end{bmatrix}, \quad
  c = \begin{bmatrix} 2 & -1 \end{bmatrix}, \quad
  d = 0,
 \end{displaymath}
 \begin{itemize}[<+-| alert@+>]
 \item[(a)] discuss its stability,
 \item[(b)] express it in scalar form,
 \item[(c)] compute its transfer function,
 \item[(d)] compute the first three values ($k=0,1,2$) of its response to
            \begin{displaymath}
             x(0) = \begin{bmatrix} 2 \\ 1 \end{bmatrix}, \quad
             u(k) = 0.4k.
            \end{displaymath}
 \end{itemize}
\end{frame}

\begin{frame}
\frametitleTC{Solution}
\framesubtitleTC{Item (a)}
\myPause
 \begin{itemize}[<+-| alert@+>]
 \item We need to compute the eigenvalues $\lambda_{1,2}$ of $A$:
       \begin{itemize}[<+-| alert@+>]
       \item[] \vspace{1mm}
               $\det(\lambda I -A) = 0$,
       \item[] \vspace{1mm}
               $\det \left(
                     \begin{bmatrix} \lambda & 0 \\ 0 & \lambda \end{bmatrix}
                    -\begin{bmatrix} 0.4 & 0.4 \\ 0.05 & 0.3 \end{bmatrix}
                     \right) = 0 $,
       \item[] \vspace{1mm}
               $\det \begin{bmatrix} \lambda-0.4 & -0.4 \\
                                    -0.05       & \lambda-0.3 \end{bmatrix} = 0 $,
       \item[] \vspace{1mm}
               $(\lambda-0.4)(\lambda-0.3)-(-0.4)(-0.05) = 0$,
       \item[] \vspace{1mm}
               $\lambda^2-0.7\lambda+0.1=0$,
       \item[] \vspace{1mm}
               $\lambda=\cfrac{0.7\mp\sqrt{0.7^2-4\cdot 0.5}}{2} \quad \Rightarrow \quad
                \lambda_1=0.2, \; \lambda_2=0.5.$
       \end{itemize}
 \item All the eigenvalues of $A$ are strictly less than one in magnitude\\
       $\Rightarrow$ the system is asymptotically stable.
 \end{itemize}
\end{frame}

\begin{frame}
\frametitleTC{Solution}
\framesubtitleTC{Item (b)}
\myPause
 \begin{itemize}[<+-| alert@+>]
 \item Denoting by $x=[x_1\;x_2]'$ the state vector (sign $'$means transpose) we have
       \begin{displaymath}
        \left\{\begin{array}{rcl}
         \begin{bmatrix} x_1(k) \\ x_2(k) \end{bmatrix} 
         &=&             
         \begin{bmatrix} 0.4 & 0.4 \\ 0.05 & 0.3 \end{bmatrix}\,
         \begin{bmatrix} x_1(k-1) \\ x_2(k-1) \end{bmatrix}
         +
         \begin{bmatrix} 1 \\ 0.5 \end{bmatrix}\,
         u(k-1) \\
         y(k)
         &=&          
         \begin{bmatrix} 2 & -1 \end{bmatrix} \,
         \begin{bmatrix} x_1(k) \\ x_2(k) \end{bmatrix}
        \end{array}\right.
       \end{displaymath}
 \item hence in scalar form
       \begin{displaymath}
        \left\{\begin{array}{rcl}
         x_1(k) &=&  0.4 x_1(k-1) + 0.4 x_2(k-1) +    u(k-1)\\
         x_2(k) &=& 0.05 x_1(k-1) + 0.3 x_2(k-1) + 0.5u(k-1)\\
         y(k)   &=&    2 x_1(k-1) -     x_2(k-1)
        \end{array}\right.
       \end{displaymath}
 \end{itemize}
\end{frame}

\begin{frame}
\frametitleTC{Solution}
\framesubtitleTC{Item (c)}
\myPause
 \begin{itemize}[<+-| alert@+>]
 \item The transfer function $G(z)$ is $c(zI-A)^{-1}b+d$, hence
       \begin{itemize}
       \item[] \begin{itemize}[<+-| alert@+>]
               \item[$G(z)$] \vspace{1mm}
                    $= \begin{bmatrix} 2 & -1 \end{bmatrix} \,
                       \begin{bmatrix} z-0.4 & -0.4 \\
                       -0.05& z-0.3 \end{bmatrix}^{-1} \,
                       \begin{bmatrix} 1 \\ 0.5 \end{bmatrix}
                       +0
                    $
               \item[] \vspace{1mm}
                    $= \cfrac{1}{(z-0.5)(z-0.2)}
                       \begin{bmatrix} 2 & -1 \end{bmatrix} \,
                       \begin{bmatrix} z-0.3 & 0.4 \\
                       0.05& z-0.4 \end{bmatrix} \,
                       \begin{bmatrix} 1 \\ 0.5 \end{bmatrix}
                    $
               \item[] \vspace{1mm}
                    $= \cfrac{1}{(z-0.5)(z-0.2)}
                       \begin{bmatrix} 2z-0.65 &-z+1.2 \end{bmatrix} \,
                       \begin{bmatrix} 1 \\ 0.5 \end{bmatrix}
                    $
               \item[] \vspace{1mm}
                    $= \cfrac{1.5z-0.05}{(z-0.5)(z-0.2)}.
                    $
               \end{itemize}
       \end{itemize}
 \end{itemize}
\end{frame}

\begin{frame}
\frametitleTC{Solution}
\framesubtitleTC{Item (d)}
\myPause
 \begin{itemize}[<+-| alert@+>]
 \item We have\\
       $u(0) = 0, \quad u(1) = 0.4, \quad u(2) = 0.8, \quad u(3) = 1.2, \, \ldots$
 \item We need to iteratively apply the state and output equations, whence
       \begin{itemize}
       \item[] \begin{itemize}[<+-| alert@+>]
               \item[$k=0$:] \vspace{1mm} 
                    $\left\{\begin{array}{rll}
                     x(0) &= \begin{bmatrix} 2 \\ 1 \end{bmatrix} (\text{given}) \\
                     y(0) &= \begin{bmatrix} 2 & -1 \end{bmatrix} \,
                             \begin{bmatrix} 2 \\ 1 \end{bmatrix}
                          &= 3
                     \end{array}\right.
                    $
               \item[$k=1$:] \vspace{1mm} 
                    $\left\{\begin{array}{rll}
                     x(1) &= \begin{bmatrix} 0.4 & 0.4 \\ 0.05 & 0.3 \end{bmatrix} \,
                             \begin{bmatrix} 2 \\ 1 \end{bmatrix}
                            +\begin{bmatrix} 1 \\ 0.5 \end{bmatrix} \cdot 0
                          &= \begin{bmatrix} 1.2 \\ 0.4 \end{bmatrix}\\
                     y(1) &= \begin{bmatrix} 2 & -1 \end{bmatrix} \,
                             \begin{bmatrix} 1.2 \\ 0.4 \end{bmatrix}
                          &= 2
                     \end{array}\right.
                    $
               \item[$k=2$:] \vspace{1mm} 
                    $\left\{\begin{array}{rll}
                     x(1) &= \begin{bmatrix} 0.4 & 0.4 \\ 0.05 & 0.3 \end{bmatrix} \,
                             \begin{bmatrix} 1.2 \\ 0.4 \end{bmatrix}
                            +\begin{bmatrix} 1 \\ 0.5 \end{bmatrix} \cdot 0.4
                          &= \begin{bmatrix} 1.04 \\ 0.38 \end{bmatrix}\\
                     y(1) &= \begin{bmatrix} 2 & -1 \end{bmatrix} \,
                             \begin{bmatrix} 1.04 \\ 0.38 \end{bmatrix}
                          &= 1.7
                     \end{array}\right.
                    $
               \end{itemize}
       \end{itemize}
 \end{itemize}
\end{frame}

\begin{frame}
\frametitleTC{Addendum}
\framesubtitleTC{}
\myPause
 \begin{itemize}[<+-| alert@+>]
 \item We define some signals useful for the following.
       \begin{itemize}[<+-| alert@+>]
       \item Impulse (precisely, \TC{unit} impulse as the value is 1):
             \begin{displaymath}
              imp(k) = \begin{cases} 1 & k=0 \\ 0 & \text{otherwise} \end{cases}
             \end{displaymath}
       \item Step (\TC{unit} step, amplitude is 1):
             \begin{displaymath}
              step(k) = \begin{cases} 1 & k \geq 0 \\ 0 & \text{otherwise} \end{cases}
             \end{displaymath}
       \item Ramp(\TC{unit} ramp, slope is 1):
             \begin{displaymath}
              ramp(k) = k\; step(k) = \begin{cases} k & k \geq 0 \\ 0 & \text{otherwise} \end{cases}
             \end{displaymath}
       \end{itemize}
 \item Quite frequently ``unit'' is omitted, e.g. ``step response'' actually\\
       means ``unit step response''.
 \end{itemize}
\end{frame}

\begin{frame}
\frametitleTC{Proposed exercise 01}
\framesubtitleTC{Try this at home, ask questions next time if needed}
\myPause
 Given the DT LTI dynamic system described in the state space by
 \begin{displaymath}
  A = \begin{bmatrix} 0.5 & 0 \\ 2 & -0.3 \end{bmatrix}, \quad
  b = \begin{bmatrix} 1 \\ 0 \end{bmatrix}, \quad
  c = \begin{bmatrix} 0 & 4 \end{bmatrix}, \quad
  d = 1,
 \end{displaymath}
 \begin{itemize}[<+-| alert@+>]
 \item[(a)] discuss its stability,
 \item[(b)] express it in scalar form,
 \item[(c)] compute its transfer function,
 \item[(d)] compute the first three values ($k=0,1,2$) of its response to
            \begin{displaymath}
             x(0) = \begin{bmatrix} 1 \\ 1 \end{bmatrix}, \quad
             u(k) = 2 step(k).
            \end{displaymath}
 \end{itemize}
\end{frame}

\begin{frame}[fragile]
\frametitleTC{But how can we check our results?}
\framesubtitleTC{}
\myPause
 \begin{itemize}[<+-| alert@+>]
 \item With the symbolic package wxMaxima:
       \begin{verbatim}
        A  : matrix([0.4,0.4],[0.05,0.3]);
        b  : matrix([1],[0.5]);
        c  : matrix([2,-1]);
        d  : 0;
        G  : factor(c.invert(z*ident(2)-A).b+d);
        Gr : rat(G,z);
        x0 : matrix([2],[1]);
        y0 : c.x0;
        x1 : A.x0+b*0;   /* the 0   is u(0) */
        y1 : c.x1;
        x2 : A.x1+b*0.4; /* the 0.4 is u(1) */
        y2 : c.x2;
       \end{verbatim}

 \end{itemize}
\end{frame}

\begin{frame}
\frametitleTC{But how can we check our results?}
\framesubtitleTC{}
\myPause
 \begin{itemize}[<+-| alert@+>]
 \item We are learning (the bit we need of) wxMaxima by example.
 \item For the moment:
       \begin{itemize}[<+-| alert@+>]
       \item you SET with \texttt{:}, \texttt{=} is for equations,
       \item \texttt{[} and \texttt{]} delimit a list,
       \item matrices are defined with \texttt{matrix} as one list per row,
       \item you multiply scalars (or by a scalar) with \texttt{*}, matrices \& vectors with \texttt{.} (period), 
       \item \texttt{ident(n)} is identity of dimension n,
       \item \texttt{invert} is self-explanatory, you also have \texttt{transpose}, \texttt{determinant},\\
             \texttt{eigenvalues}, \texttt{eigenvectors} and much more,
       \item \texttt{factor} attempts to factor an expression, \texttt{rat(expr,var)} to express\\
             \texttt{expr} rationally wrt \texttt{var};
       \item enjoy \smiley...
       \end{itemize}
 \end{itemize}
\end{frame}

\begin{frame}
\frametitleTC{Takeaways}
\framesubtitleTC{from exercise 01 (and the proposed one)}
\myPause
 \begin{itemize}[<+-| alert@+>]
 \item A transfer function is the ratio of two polynomials.
 \item We call the roots of its numerator the \TC{zeroes}.
 \item We call the roots of its denominator the \TC{poles}. 
 \item The poles are eigenvalues of $A$.
 \item The degree of the numerator is at most equal to that of the denominator,
 \item and equal iff $d \neq 0$ (you will see this in the proposed exercise, try to\\
       prove it holds true in general).
 \item We call the number of poles minus that of zeroes the \TC{relative degree}\\
       of the system.
 \end{itemize}
\end{frame}

