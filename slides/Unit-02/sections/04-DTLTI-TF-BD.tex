\section{DT LTI -- transfer function and block diagrams}

\subsection{}

\begin{frame}
\frametitleTC{Preliminary}
\framesubtitleTC{for an alternative system representation, particularly useful for control}
\myPause
 \begin{itemize}[<+-| alert@+>]
 \item Let us introduce the \TC{one-step advance operator} $z$:
       \begin{displaymath}
        z\nu(k) := \nu(k+1) \qquad\qquad \text{whatever $\nu$ is}.
       \end{displaymath}
 \item This allows for a compact way to write LTI systems without evidencing the state (but it is there!).
 \item Suppose for example that we have a system with input $u$ and output $y$, ruled by
       \begin{displaymath}
        y(k) = ay(k-1)+bu(k-1),
       \end{displaymath}
       where clearly the state is the previous value of $y$;
 \item we can re-write $y(k+1)=ay(k)+bu(k)$ in the form
       \begin{displaymath}
        (z-a)y(k) = bu(k).
       \end{displaymath}
 \end{itemize}
\end{frame}

\begin{frame}
\frametitleTC{Preliminary}
\framesubtitleTC{introducing the transfer function}
\myPause
 \begin{itemize}[<+-| alert@+>]
 \item Rearranging -- remember the \emph{operatorial} meaning of $z$ -- we get
       \begin{displaymath}
        y(k) = \frac{b}{z-a}u(k)
       \end{displaymath}
\item[] and we can define the \TC{Transfer Function (TF)} notation, that expresses the system\\
        as a compound operator built upon the elementary one $z$:
       \begin{itemize}[<+-| alert@+>]
       \item we write the system's transfer function $G(z)$ as
             \begin{displaymath}
               G(z) = \frac{b}{z-a},
             \end{displaymath}
       \item[] and in force of this
             \begin{displaymath}
              y(k) = G(z)u(k) \quad \text{means} \quad y(k) = ay(k-1)+bu(k-1).
             \end{displaymath}
       \end{itemize}
 \item \vfill May not look so useful at the moment, but you will see later on\\
       how handy transfer functions are to determine controllers.
 \end{itemize}
\end{frame}

\begin{frame}
\frametitleTC{State space representation and transfer function}
\framesubtitleTC{}
\myPause
 \begin{itemize}[<+-| alert@+>]
 \item Given the SS representation we saw, shift all times by one (which is legit as the system is TI):
       \begin{displaymath}
        \left\{
         \begin{array}{rlll}
          x(k+1) &= A x(k)   + b u(k)\\
          y(k+1) &= c x(k+1) + d u(k+1) 
         \end{array}
        \right.
       \end{displaymath} 
 \item Apply the advance operator:
       \begin{displaymath}
        \left\{
         \begin{array}{rl}
          \textcolor{red}{z}x(k) &= A x(k) + b u(k)\\
          \textcolor{red}{z}y(k) &= c \textcolor{red}{z}x(k) + d\textcolor{red}{z} u(k)
         \end{array}
        \right.
       \end{displaymath} 
 \item Drop useless $z$'s and rearrange ($I$ is identity matrix of dimension $n$):
       \begin{displaymath}
        \left\{
         \begin{array}{rl}
          (zI-A)x(k) &= b u(k)\\
          y(k)       &= c x(k) + d u(k)
         \end{array}
        \right.
       \end{displaymath} 
 \end{itemize}
\end{frame}

\begin{frame}
\frametitleTC{State space representation and transfer function}
\framesubtitleTC{once again, sticking to the SISO case}
\myPause
 \begin{itemize}[<+-| alert@+>]
 \item Solve state equation for $x(k)$:
       \begin{displaymath}
        x(k) = (zI-A)^{-1} b u(k)
       \end{displaymath} 
 \item Substitute into output equation and rearrange:
       \begin{displaymath}
        y(k) = \textcolor{red}
                         {\underbrace{\left[c(zI-A)^{-1} b +d \right]}
                                    _{\text{Transfer function } G(z)}
                         } \; u(k)
       \end{displaymath} 
 \item We then define the transfer function of the generic SISO DT LTI\\
       system $(A,b,c,d)$ as
       \begin{displaymath}
        G(z) = c(zI-A)^{-1} b +d.
       \end{displaymath} 
 \end{itemize}
\end{frame}

\begin{frame}
\frametitleTC{Block diagrams}
\framesubtitleTC{Preliminaries}
\myPause
\begin{itemize}[<+-| alert@+>]
\item Block diagrams (BDs) are a graphical formalism to represent dynamic systems, that we see here limited to the
      DT LTI class.
\item They are useful to study \TC{interconnected} systems, i.e., compounds of \TC{subsystems} (e.g., a controller
      and the controlled object).
\item Important \emph{caveat}, that we state right from the beginning:
      \begin{itemize}[<+-| alert@+>]
      \item    a BD MUST NOT BE CONFUSED with a flow diagram;
      \item [] although subsystems have inputs and outputs,
      \item [] their compound comes from assembling \TC{equations}.
      \end{itemize}
\end{itemize}
\end{frame}

\begin{frame}
\frametitleTC{Block diagram components}
\framesubtitleTC{Block and summation node}
\myPause
\begin{center}
 \begin{picture}(210,70)(0,0)
  \put(000,35){\vector(1,0){30}}
  \put(000,40){{\small $u(k)$}}
  \put(030,20){\framebox(50,30)
              {$G(z)$}
              }
  \put(080,35){\vector(1,0){30}}
  \put(090,40){{\small $y(k)$}}

  \put(175,35){\circle{10}}

  \put(175,70){\vector(0,-1){30}}
  \put(150,62){{\small $u_1(k)$}}
  \put(165,40){{\small $+$}}

  \put(140,35){\vector(1,0){30}}
  \put(140,40){{\small $u_2(k)$}}
  \put(162,27){{\small $-$}}

  \put(175,00){\vector(0,1){30}}
  \put(150,03){{\small $u_3(k)$}}
  \put(167,20){{\small $+$}}

  \put(180,35){\vector(1,0){30}}
  \put(190,40){{\small $y(k)$}}
\end{picture}

\end{center} \myPause
\begin{itemize}[<+-| alert@+>]
\item \TC{Block} (left):\\
      an LTI SISO system with the indicated transfer function -- in the shown\\
      example, the \TC{equation} $G(z)u(k)-y(k)=0$.
\item \TC{Summation node} (right):\\
      a summation expression -- in the shown example, the \TC{equation}
      $u_1(k)-u_2(k)+u_3(k)-y(k)=0$.
\end{itemize}
\end{frame}

