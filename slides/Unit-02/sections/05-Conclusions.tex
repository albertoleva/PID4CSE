\section{Conclusions}
\subsection{}

\begin{frame}
\frametitleTC{Conclusions}
\framesubtitleTC{Recap and lessons learnt (1/2)}
\myPause
 \begin{itemize}[<+-| alert@+>]
 \item Feedback can counteract model errors and disturbances, even without measuring them.
 \item This is a particularly counter-intuitive idea, but one of the major strengths of control: keep this in mind.
 \item Feedback can stabilise an unstable system...
 \item ...or more precisely, close a loop around an unstable system so that the \TC{compound} be stable.
 \item A dynamic system can be represented in \TC{state space} form.
 \item One can compute equilibria and motions from that representation.
 \item Equilibria can be \TC{asymptotically stable}, \TC{stable} or \TC{unstable}.
 \end{itemize}
\end{frame}

\begin{frame}
\frametitleTC{Conclusions}
\framesubtitleTC{Recap and lessons learnt (2/2)}
\myPause
 \begin{itemize}[<+-| alert@+>]
 \item In the LTI case, the initial state and the input combine their effects linearly.
 \item In the LTI case, stability is a property of the system.
 \item A LTI system is represented in state space form as $(A,b,c,d)$.
 \item Its stability only depends on the eigenvalues of $A$.
 \item By introducing the \TC{one-step advance operator} $z$ we can write a DI LTI system\\
       in \TC{transfer function} form.
 \item Transfer functions can be used as blocks in \TC{block diagrams}\\
       to represent systems made of interconnected subsystems.
 \end{itemize}
\end{frame}

\begin{frame}
\frametitleTC{Conclusions}
\framesubtitleTC{Next step}
\myPause
 \begin{itemize}[<+-| alert@+>]
 \item Go through a practice session
       \begin{itemize}[<+-| alert@+>]
       \item to play around with dynamic systems, mostly DT LTI,
       \item and start to get in touch with useful software.
       \end{itemize}
 \item \vfill Your assignments prior to that session:
       \begin{itemize}[<+-| alert@+>]
       \item positively review your notes and ask questions at the beginning of the session,\\
             otherwise it will be practically useless;
       \item install Scilab (\texttt{www.scilab.org}),\\
             OpenModelica (\texttt{www.openmodelica.org}),\\
             and wxMaxima (\texttt{wxmaxima-developers.github.io/wxmaxima}).
       \end{itemize}
 \end{itemize}
\end{frame}

\section{}
{
\setbeamertemplate{headline}{
  \begin{beamercolorbox}[wd=\paperwidth,ht=4.2ex,dp=1.5ex]{palette quaternary}
  \end{beamercolorbox}
  }
\setbeamertemplate{footline}{
  \begin{beamercolorbox}[wd=\paperwidth,ht=2.2ex,dp=1.5ex]{palette quaternary}
  \end{beamercolorbox}
  }
\begin{frame}[noframenumbering]
 \vspace{20mm}\Huge{Discussion open}
\end{frame}
}

