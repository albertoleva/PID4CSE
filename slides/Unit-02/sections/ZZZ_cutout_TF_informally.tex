\begin{frame}
\frametitleTC{Informal overview of our main points about feedback}
\framesubtitleTC{i.e., let us grasp intuitively ideas that we are then going to treat mathematically}
\myPause
 \begin{itemize}[<+-| alert@+>]
 \item Preliminary -- let us introduce (for the moment, informally) the \TC{one-step advance operator} $z$:
       \begin{displaymath}
        z\nu(k) := \nu(k+1) \qquad\qquad \text{whatever $\nu$ is}.
       \end{displaymath}
 \item This allows for a compact way to write LTI systems without evidencing the state (but it is there!).
 \item Suppose for example that we have a system with input $u$ and output $y$, ruled by
       \begin{displaymath}
        y(k) = ay(k-1)+bu(k-1),
       \end{displaymath}
       where clearly the state is the previous value of $y$;
 \item we can re-write $y(k+1)=ay(k)+bu(k)$ in the form
       \begin{displaymath}
        (z-a)y(k) = bu(k).
       \end{displaymath}
 \end{itemize}
\end{frame}

\begin{frame}
\frametitleTC{Informal overview of our main points about feedback}
\framesubtitleTC{}
\myPause
 \begin{itemize}[<+-| alert@+>]
 \item Rearranging -- remember the \emph{operatorial} meaning of $z$ -- we get
       \begin{displaymath}
        y(k) = \frac{b}{z-a}u(k)
       \end{displaymath}
\item[] and we can define the \TC{transfer function} notation (rigorous math on this later on),\\
       that expresses the system as an operator:
       \begin{itemize}[<+-| alert@+>]
       \item we set
             \begin{displaymath}
               G(z) = \frac{b}{z-a},
             \end{displaymath}
       \item[] and in force of this
             \begin{displaymath}
              y(k) = G(z)u(k) \quad \text{means} \quad y(k) = ay(k-1)+bu(k-1).
             \end{displaymath}
       \end{itemize}
 \item \vfill May not look so useful at the moment, but you will see later on\\
       how handy transfer functions are to determine controllers.
 \end{itemize}
\end{frame}
