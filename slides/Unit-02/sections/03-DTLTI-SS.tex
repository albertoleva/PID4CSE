\section{DT LTI -- state space}

\subsection{}

\begin{frame}
\frametitleTC{The DT LTI class}
\framesubtitleTC{Preliminaries}
\begin{itemize}[<+-| alert@+>]
\item For the LTI class, that we see here almost exclusively in the DT context, there exists a very strong theory.
\item The same is not true for more general (e.g., nonlinear) system classes.
\item Therefore, control problems of the type addressed here, are cast in the LTI framework wherever possible.
\item \vfill This motivates the importance of the DT LTI class, which we now\\
      come to examine.
\end{itemize}
\end{frame}

\begin{frame}
\frametitleTC{State-space representation of a DT LTI system}
\framesubtitleTC{Definition --- we stick to the SISO case for simplicity}
\myPause
\begin{itemize}[<+-| alert@+>]
\item The \TC{State-Space (SS)} representation of a DT LTI dynamic system is the quadruplet $(A,b,c,d)$ in
      \begin{displaymath}
       \left\{
        \begin{array}{rlll}
         x(k) &= A x(k-1) + b u(k-1) & & \text{\TC{(state equation)}}  \\
         y(k) &= c x(k)   + d u(k)   & & \text{\TC{(output equation)}}
        \end{array}
       \right.
      \end{displaymath}    
\item $u(k)$ and $y(k)$ are real scalars;
\item $x(k) \in \Re^n$, where $n$ is the system's order;
\item the real \TC{dynamic matrix} $A$ is $n \times n$;
\item $b$ is a real column vector ($n \times 1$);
\item $c$ is a real row vector ($1 \times n$);
\item $d$ is a real scalar.
\end{itemize}
\end{frame}

\begin{frame}
\frametitleTC{Motion}
\framesubtitleTC{}
\myPause
\begin{itemize}[<+-| alert@+>]
\item Given $x(0)$ and $u(k)$, $k \geq 0$, we get
      {\small
      \begin{displaymath}
       \begin{array}{rclcl}
        x(1) &=& Ax(0)+bu(0) \\
        x(2) &=& Ax(1)+bu(1) &=& A^2x(0)+Abu(0)+bu(1) \\
        x(3) &=& Ax(2)+bu(2) &=& A^3x(0)+A^2bu(0)+Abu(1)+bu(2) \\
             & & \cdots\\
       \end{array}
      \end{displaymath}
      }
\item This readily generalises to the \TC{Lagrange state formula}
      \begin{displaymath}
       x(k) = A^k x(0) +\sum\limits_{h=0}^{k-1} A^{k-h-1}bu(h).
      \end{displaymath}
\end{itemize}
\end{frame}

\begin{frame}
\frametitleTC{Free and induced state motion}
\framesubtitleTC{}
\myPause
\begin{itemize}[<+-| alert@+>]
\item In LTI systems, the state motion $x(k)$ is the \TC{sum} of a \TC{free motion} $x_F(k)$ and an
      \TC{induced motion} $x_I(k)$, i.e.,
      \begin{displaymath}
       x(k) = x_F(k) + x_I(k),
      \end{displaymath} \myPause
      where
      \begin{displaymath}
       x_F(k) = A^k x(0), \quad
       x_I(k) = \sum\limits_{h=0}^{k-1} A^{k-h-1}bu(h).
      \end{displaymath} \myPause
\item $x_F(k)$ depends linearly on $x(0)$ and not on $u$,
\item while $x_I(k)$ depends linearly only on $u(k)$ and not on $x(0)$.
\end{itemize}
\end{frame}

\begin{frame}
\frametitleTC{Free and induced output motion}
\framesubtitleTC{}
\myPause
\begin{itemize}[<+-| alert@+>]
\item In LTI systems, also the state motion $y(k)$ is the \TC{sum} of a \TC{free motion} $y_F(k)$ and an
      \TC{induced motion} $y_I(k)$, i.e.,
      \begin{displaymath}
       y(k) = y_F(k) + y_I(k),
      \end{displaymath} \myPause
      where
      \begin{displaymath}
       y_F(k) = cA^k x(0), \quad
       y_I(k) = c\sum\limits_{h=0}^{k-1} A^{k-h-1}bu(h) +du(k).
      \end{displaymath} \myPause
\item again, $y_F(k)$ depends linearly on $x(0)$ and not on $u$,
\item while $y_I(k)$ depends linearly only on $u(k)$ and not on $x(0)$.
\end{itemize}
\end{frame}

\begin{frame}
\frametitleTC{Equilibrium}
\framesubtitleTC{}
\myPause
\begin{itemize}[<+-| alert@+>]
\item For an LTI system, equilibrium states are found by solving
      \begin{displaymath}
       \overline{x} = A\overline{x}+b\overline{u}.
      \end{displaymath}
\item Thus, if $A$ has no unity eigenvalues, there exists the one equilibrium
      \begin{displaymath}
       \overline{x} = (I-A)^{-1}b\overline{u},
      \end{displaymath}
\item while in the opposite case, either there is no equilibrium, or there\\
      are infinite ones.
\end{itemize}
\end{frame}

\begin{frame}
\frametitleTC{Equilibrium}
\framesubtitleTC{Peculiarities of L(TI) systems}
\myPause
\begin{itemize}[<+-| alert@+>]
\item Contrary to nonlinear systems, they cannot have a finite number of equilibria, different from zero and one.
\item If some $\overline{u}$ produces zero, one or infinite equilibria, the same is true for any other $\overline{u}$,
      contrary again to the nonlinear case.
\item For each equilibrium state, there surely exists the one equilibrium output
      \begin{displaymath}
       \overline{y}=c\overline{x}+d\overline{u}.
      \end{displaymath}
\end{itemize}
\end{frame}

\begin{frame}
\frametitleTC{Stability of an equilibrium}
\framesubtitleTC{}
\myPause
\begin{itemize}[<+-| alert@+>]
\item Let $(\overline{x},\overline{u})$ be an equilibrium for an LTI system. 
\item The Lagrange state formula leads to write
      \begin{displaymath}
       \overline{x} = A^k \overline{x} +\sum\limits_{h=0}^{k-1} A^{k-h-1}b\overline{u}.
      \end{displaymath}.
\item Consider now the  perturbed motion $x_{\Delta}(k)$ produced by $\overline{u}$ and
      $x(0)=\overline{x}+\Delta\overline{x}$; the same formula yields
      \begin{displaymath}
       x_{\Delta}(k) = A^k (\overline{x}+\Delta\overline{x}) +\sum\limits_{h=0}^{k-1} A^{k-h-1}b\overline{u}.
      \end{displaymath}
\item Subtracting, therefore,
      \begin{displaymath}
       x_{\Delta}(k)-\overline{x} = A^k\Delta\overline{x}
      \end{displaymath}
\end{itemize}
\end{frame}

\begin{frame}
\frametitleTC{Stability of an equilibrium}
\framesubtitleTC{and in the L(TI) case, of a system}
\myPause
\begin{itemize}[<+-| alert@+>]
\item The way $x_{\Delta}(k)$ moves with respect to $\overline{x}$ does not depend on $\overline{x}$.
\item That is, contrary the nonlinear case, there cannot be equilibria with different stability characteristics.
\item In the L(TI) class, stability is a property of the \emph{system}, not of the individual equilibria.
\item Moreover, 
      \begin{itemize}[<+-| alert@+>]
      \item for $k\rightarrow\infty$, $||x_{\Delta}(k)-\overline{x}|| \rightarrow 0 \, \forall x(0)$  iff $A^k$
            converges to a zero matrix,
      \item the same norm generally diverges if at least one element of $A^k$ does,
      \item and if $A^k$ neither converges to zero nor diverges, the same happens\\
            to $||x_{\Delta}(k)-\overline{x}||$.
      \end{itemize}
\item Thus, in the LTI case, the stability of a system only depends on its\\
      dynamic matrix $A$.
\end{itemize}
\end{frame}

\begin{frame}[label={pag:stab-eivals}]
\frametitleTC{Stability and eigenvalues of $A$}
\framesubtitleTC{}
\myPause
\begin{itemize}[<+-| alert@+>]
\item The following can be proven.
      \begin{itemize}[<+-| alert@+>]
      \item An LTI system is asymptotically stable iff all the eigenvalues of $A$ have magnitude less  than one
            (or, equivalently, lie in the open \TC{unit circle} of the complex plane).
      \item The same system is unstable if (but not only if) at least one eigenvalue of $A$ has magnitude greater
            than one.
      \item If all the eigenvalues of $A$ have magnitude less than or equal to one, and there exists at least one
            with unity magnitude, the system can be either unstable\\
            or stable, but not asymptotically.
      \end{itemize}
\end{itemize}
\end{frame}

\begin{frame}
\frametitleTC{Properties of asymptotically stable systems}
\framesubtitleTC{also in a view to control}
\myPause
\begin{itemize}[<+-| alert@+>]
\item An asymptotically stable system has one and only one equilibrium for each constant input.
\item The state and output free motions of an asymptotically stable system converge to zero (norm)
      for $k\rightarrow\infty$.
\item As a consequence, asymptotically stable systems ``forget their initial condition''...
\item ...which is a definitely desired property for a \TC{controlled} system.
\end{itemize}
\end{frame}

