\section{Conclusions}
\subsection{}

\begin{frame}
\frametitleTC{Conclusions}
\framesubtitleTC{Recap and lessons learnt (1/2)}
\myPause
 \begin{itemize}[<+-| alert@+>]
 \item We can characerise a dynamic system with \TC{time domain responses}.
 \item These refer to a certain \emph{stimulus} (we used impulse and step).
 \item Responses can be used also to stipulate \TC{control objectives}.
 \item A viable way to do so is to transform time domain deires into some desired
       closed-loop transfer function.
 \item This paves the way to the \TC{direct synthesis} technique.
 \end{itemize}
\end{frame}

\begin{frame}
\frametitleTC{Conclusions}
\framesubtitleTC{Recap and lessons learnt (2/2)}
\myPause
 \begin{itemize}[<+-| alert@+>]
 \item We have open issues, however.
       \begin{itemize}[<+-| alert@+>]
       \item Why impulse and step \emph{stimuli} are meaningful?
       \item How can we choose the target closed-loop transfer function?
       \item What are the limits and what is their origin?
       \end{itemize}
 \item \vfill These issues are very suited for addressing in a practice session,\\
       which is our next task.
 \item Please review your notes, next time ask questions if needed\\
       before we start.
 \end{itemize}
\end{frame}

\section{}
{
\setbeamertemplate{headline}{
  \begin{beamercolorbox}[wd=\paperwidth,ht=4.2ex,dp=1.5ex]{palette quaternary}
  \end{beamercolorbox}
  }
\setbeamertemplate{footline}{
  \begin{beamercolorbox}[wd=\paperwidth,ht=2.2ex,dp=1.5ex]{palette quaternary}
  \end{beamercolorbox}
  }
\begin{frame}[noframenumbering]
 \vspace{20mm}\Huge{Discussion open}
\end{frame}
}

