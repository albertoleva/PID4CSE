\section{}
{
\setbeamertemplate{headline}{
  \begin{beamercolorbox}[wd=\paperwidth,ht=4.2ex,dp=1.5ex]{palette quaternary}
  \end{beamercolorbox}
  }
\setbeamertemplate{footline}{
  \begin{beamercolorbox}[wd=\paperwidth,ht=2.2ex,dp=1.5ex]{palette quaternary}
  \end{beamercolorbox}
  }
\begin{frame}[noframenumbering]
 \vspace{20mm}\Huge{Thank you for attending}\\
 \vspace{20mm}\large{\texttt{alberto.leva@polimi.it}}
\end{frame}
}

\begin{frame}[plain,noframenumbering]
 COPYRIGHT AND LICENCE NOTES \\
 {\tiny

\vspace{3mm}The slides in this document are part of a project titled PID4CSE (PID control for Computer Scientists and Engineers). The project public repository is \texttt{https://github.com/albertoleva/PID4CSE}.\par

\vspace{1mm}The PID4CSE project consists of slides, Scilab and wxMaxima scripts, and Modelica models; excerpts from the said scripts and models are also reported in the slides.\par
 
\vspace{1mm}The material of PID4CSE, as noted at the beginning for these slides, is released under a CC-BY-SA 4.0 International licence. The same licence as the slides applies to scripts and models. Exceptions are however logos, and some of the images contained in the slides.\par

\vspace{1mm}Logos -- for example, the one of the Politecnico di Milano that the author includes in his published version of the slides as a tribute to his \emph{Alma Mater} but is not contained as a separate file in the PID4CSE distribution -- are the property of their owners and holders, like any mentioned third party mark or brand. As for the images (not logos) in the slides, most of them were produced from scratch by the author, and are released within the CC-BY-SA 4.0 licence. Some others were instead either downloaded from the web, or composed by assembling and/or editing downloaded images or parts of them. For images of this second type, care was taken to use only material released under CC0 or substantially equivalent licences, paying special attention to verify that the said licences (i) require no attribution and (ii) permit commercial usage. so as to allow the overall work to be distributed within a ''free culture'' licence (in the case you notice that a mistake was made in this respect, please take into account that this happened in good faith, notify the author by writing to \texttt{alberto.leva@polimi.it}, and the convenient remedial action will be taken as soon as possible).\par

\vspace{1mm}Images of the second type defined above, are licensed under a Creative Commons CC0 1.0 Universal licence. In the PID4CSE repository, the files for these images are identified by appending a ``\texttt{\_cc0}'' suffix to their name, before the extension. For example, \texttt{foo.jpg} is an image of the first type and is CC-BY-SA, while \texttt{bar\_cc0.png} belongs to the second type and is CC0. To identify images of the second type (CC0) also in this pdf document, the slides that contain them are marked with a small coloured square (\TC{$\blacksquare$}) in the upper right corner of the canvas, just below the section headlines and slide progress bar; all the images in any such slide are therefore CC0, while all the images in the other slides (without \TC{$\blacksquare$}) are CC-BY-SA; of course, no slide mixes images of the two types.\par

\vspace{1mm}Both the used licences are in the ``licences'' folder of the PID4CSE project repository. If you are curious about the choice of the adopted licensing scheme and its motivations, please refer to the \texttt{LICENCE-NOTES.TXT} document in the same folder.\par
 }
\end{frame}
