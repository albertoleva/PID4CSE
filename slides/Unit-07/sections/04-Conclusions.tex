\section{Conclusions}
\subsection{}

\begin{frame}
\frametitleTC{Conclusions}
\framesubtitleTC{Recap and lessons learnt}
\myPause
 \begin{itemize}[<+-| alert@+>]
 \item Observing and interpreting the P, I and D actions.
 \item Relating the wide applicability of PID control to the wide representativeness of second order models.
 \item Retaining that
       \begin{itemize}[<+-| alert@+>]
       \item \TC{the structure of a control law is dictated by that of the process dynamics
             and not only by the desired closed-loop behaviour},
       \item and \TC{if the structure is properly chosen, setting parameters follows quite\\
             naturally} (while otherwise it can be trial-and-error to eternity),
       \item hence in the absence of some Systems and Control theory knowledge,\\
             there is simply no way out (if not in trivial cases).      
       \end{itemize}
 \end{itemize}
\end{frame}

\begin{frame}
\frametitleTC{Conclusions}
\framesubtitleTC{Recap and lessons learnt}
\myPause
 \begin{itemize}[<+-| alert@+>]
 \item The basics about the timeline of a control loop and the related problems.
 \item Accounting properly for (some) algorithm-related facts not shown in the LTI context, namely
       \begin{itemize}[<+-| alert@+>]
       \item positional vs. incremental form,
       \item and antiwindup type.
       \end{itemize}
 \item Understanding the potential effect of the said facts, and making\\
       knowledgeable choices so as to properly bridge system-level\\
       and implementation-level design.
 \end{itemize}
\end{frame}

\section{}
{
\setbeamertemplate{headline}{
  \begin{beamercolorbox}[wd=\paperwidth,ht=4.2ex,dp=1.5ex]{palette quaternary}
  \end{beamercolorbox}
  }
\setbeamertemplate{footline}{
  \begin{beamercolorbox}[wd=\paperwidth,ht=2.2ex,dp=1.5ex]{palette quaternary}
  \end{beamercolorbox}
  }
\begin{frame}[noframenumbering]
 \vspace{20mm}\Huge{Discussion open}
\end{frame}
}

