\section{Exercise 04}
\subsection{}

\begin{frame}
\frametitleTC{PI tuning}
\framesubtitleTC{part 1}
\myPause
 \begin{itemize}[<+-| alert@+>]
 \item Given the first-order process
       \begin{displaymath}
        P(z) = \frac{2}{(z-0.95)}
       \end{displaymath}
 \item[(a)] tune a PI as we just discussed for two/three response speeds,
 \item[(b)] test the results in Modelica.
 \end{itemize}
\end{frame}

\begin{frame}
\frametitleTC{PI tuning}
\framesubtitleTC{part 2}
\myPause
 \begin{itemize}[<+-| alert@+>]
 \item Given the processes
       \begin{displaymath}
        P_1(z) = \frac{2}{(z-0.95)(z-0.1)}, \quad
        P_2(z) = \frac{2}{(z-0.95)(z-0.8)},
       \end{displaymath}
 \item[(a)] plot their step responses and say which one is more ``dominantly first-order'',
 \item[(b)] relate your conclusions to the position of the pole not in $z=0.95$,
 \item[(c)] tune a PI for both systems as if they were first-order with only the\\
            slower pole $z=0.95$,
 \item[(b)] test the results in Modelica and comment on the possibility of\\
            reaching ``fast'' responses, depending on the position of the\\
            neglected pole.
 \end{itemize}
\end{frame}
