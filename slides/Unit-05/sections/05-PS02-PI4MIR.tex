\section{PI for the man in the road}
\subsection{}

\begin{frame}
\frametitleTC{Foreword}
\framesubtitleTC{PI (Proportional plus Integral) control as an intuitive idea}
\myPause
 \begin{itemize}[<+-| alert@+>]
 \item We now introduce PI control in a totally intuitive manner, ie., the way\\
       ``the man in the road'' would do --- and incidentally, the way somebody\\
       says it was initially invented (we are not discussing this).
 \item The purpose of this part of our activity is twofold:
       \begin{itemize}[<+-| alert@+>]
       \item show that PI control is intuitive indeed,
       \item but also that intuition without theory is often misleading.    
       \end{itemize}
 \item Exercise: as we proceed, try to spot the flaws (not necessarily true ``errors''\\
       but symptoms of a partial viewpoint) in the reasoning by the man\\
       in the road and take note of them...
 \item[] \vspace{-0.75mm}...and then, when we re-visit the matter formally later on, check\\
       your notes to see if you spotted \underline{all} such flaws.
 \end{itemize}
\end{frame}

\begin{frame}
\frametitleTC{Intuition \# 1}
\framesubtitleTC{Proportional control}
\myPause
 \begin{itemize}[<+-| alert@+>]
 \item The man in the road says: \blue{``the farther the controlled variable $y$ is\\
       from the reference $w$, the more intense the control action $u$ has to be''}.
 \item \vfill Naming $e$ the error $w-y$, this introduces the \TC{Proportional (P) action}
       \begin{displaymath}
        u_P(k) = K_P \, e(k),
       \end{displaymath}
       where $K_P$ is a configuration parameter.
 \item Seems definitely a good idea.
 \item However $u_P$ is nonzero only if so is $e$, hence with P action alone
       \begin{itemize}[<+-| alert@+>]
       \item either $y$ can stay at ANY value ($w$ can be anything) with $u=0$,
       \item or one has to accept an error to have a control action.    
       \end{itemize} 
 \end{itemize}
\end{frame}

\begin{frame}
\frametitleTC{Intuition \# 2}
\framesubtitleTC{Introducing an automatic bias}
\myPause
 \begin{itemize}[<+-| alert@+>]
 \item Alternatively, a bias can be included in $u$ so as to zero the error:
       \begin{displaymath}
        u(k) = u_P(k)+ u_{bias}.
       \end{displaymath}
 \item But how to compute $u_{bias}$?
 \item The man in the road says: \blue{``automatically; if the error is zero keep it constant\\
       because you found the right value, otherwise increase or decrease it at each step,\\
       proportionally to the error''}.    
 \item This means
       \begin{displaymath}
        u_{bias}(k) = u_{bias}(k-1)+ K_{bias} e(k),
       \end{displaymath}
       where $K_{bias}$ is another configuration parameter.
 \end{itemize}
\end{frame}

\begin{frame}
\frametitleTC{Intuition \# 3}
\framesubtitleTC{Toward integral control}
\myPause
 \begin{itemize}[<+-| alert@+>]
 \item The main in the road continues: \blue{``the automatic bias is also consistent with the\\
       idea that if the error does not diminish this means that the system is reluctant\\
       to obey to the control action, hence the said action has to become stronger\\
       and stronger''}. 
 \item In fact, $u_{bias}(k)$ as just defined, sums each new error weighed by $K_{bias}$, which comes
       to determine the ratio of the control variation \TC{rate} to the value of the error.
 \item We can say that this is \TC{integrating} the error, and name the corresponding control component
       the \TC{Integral (I) action}
       \begin{displaymath}
        u_I(k) = u_I(k-1)+ K_I e(k),
       \end{displaymath}
        with $K_I$ as the second controller configuration parameter besides $K_P$.
 \end{itemize}
\end{frame}

\begin{frame}
\frametitleTC{The PI control law}
\framesubtitleTC{}
\myPause
 \begin{itemize}[<+-| alert@+>]
 \item The control signal is the sum of the P and the I action:
       \begin{displaymath}
        \left\{\begin{array}{rcl}
         u_P(k) &=& K_P \, e(k) \\
         u_I(k) &=& u_I(k-1)+ K_I e(k) \\
         u(k)   &=& u_P(k)+u_I(k)
        \end{array}\right.
       \end{displaymath}
 \item \vfill Intuition basically stops here.
 \item To understand how to give a value to $K_P$ and $K_I$ (or to the more\\
       comfortable equivalent parameters we shall define in the following)\\
       we must resort to theory.
 \end{itemize}
\end{frame}
