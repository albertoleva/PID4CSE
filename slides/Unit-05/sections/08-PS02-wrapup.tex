\section{Wrap-up}
\subsection{}

\begin{frame}
\frametitleTC{Ideas and capabilities to take home}
\framesubtitleTC{(1/2) not necessarily in the same order as we saw them...}
 \begin{itemize}[<+-| alert@+>]
 \item Selecting $G_{yw}^{\circ}(z)$ and $G_{yd}^{\circ}(z)$ ideally and dealing
       with limitations:
       \begin{itemize}[<+-| alert@+>]
       \item no \TC{critical} (not in the unit circle) cancellations,
       \item realisable $C(z)$. 
       \end{itemize}
 \item Computing and plotting responses in Scilab, using SciNotes.
 \item Building and simulating a block diagram in Modelica with OMEdit.
 \item Applying \TC{direct synthesis} for set point tracking and disturbance\\
       rejection.
 \end{itemize}
\end{frame}

\begin{frame}
\frametitleTC{Ideas and capabilities to take home}
\framesubtitleTC{(2/2)}
 \begin{itemize}[<+-| alert@+>]
 \item Intuitive idea of PI control:
       \begin{itemize}[<+-| alert@+>]
       \item P action to react \TC{proportionally and promptly} to the error;
       \item I action to \TC{achieve zero error} at \TC{steady state} (constant inputs).
       \end{itemize}
 \item Started to analyse PI control formally, and devised a tuning rule.
 \item Initiated a discussion on when PI control is adequate to a problem.
 \item \vfill Next steps:
       \begin{itemize}[<+-| alert@+>]
       \item complete the theoretical analysis we started;
       \item introduce the D action;
       \item turn a PI(D) transfer function into a control \emph{algorithm}.
       \end{itemize}
 \end{itemize}
\end{frame}


% next next: soe typical schemes like override, some computer-related problems

