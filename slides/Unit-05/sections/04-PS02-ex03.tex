\section{Exercise 03}
\subsection{}

\begin{frame}
\frametitleTC{Direct synthesis for disturbance rejection}
\framesubtitleTC{In fact, slightly more than a remark}
\myPause
 \begin{itemize}[<+-| alert@+>]
 \item The loop  and the closed-loop transfer functions are tied to one another: since
       \begin{displaymath}
        G_{yw}(z) = \frac{L(z)}{1+L(z)}, \qquad
        G_{yd}(z) = \frac{1}{1+L(z)},
       \end{displaymath}
 \item we have
       \begin{displaymath}
         L(z) = \frac{G_{yw}(z)}{1-G_{yw}(z)}, \quad
         L(z) = \frac{1-G_{yd}(z)}{G_{yd}(z)}, \quad
         G_{yw}(z)+G_{yd}(z) = 1.
       \end{displaymath}
 \item As a consequence,
       \begin{displaymath}
        G_{yw}(z) = \frac{1-\alpha}{z-\alpha} \quad \Rightarrow \quad
        G_{yd}(z) = 1-\frac{1-\alpha}{z-\alpha}
                  = \frac{z-1}{z-\alpha}
       \end{displaymath}
       and we are just back to the tracking problem.
 \end{itemize}
\end{frame}

\begin{frame}
\frametitleTC{Direct synthesis for disturbance rejection}
\framesubtitleTC{In fact, slightly more than a remark}
\myPause
 \begin{itemize}[<+-| alert@+>]
 \item We can revisit the previous exercise focusing on the disturbance response.
 \item In doing so, it is interesting to also look at the behaviour of the \TC{control signal}\\
       (in our Modelica diagram, \texttt{C.y}).
 \item Observe how a faster rejection -- like a faster tracking -- requires a larger peak\\
       in that signal (which is quite intuitive).
 \item As we shall see, this may lead to require something that the system -- or more\\
       precisely, the \TC{actuator} -- cannot do (e.g., allocate more cores than those\\
       aboard the machine).
 \item Incidentally, this means that \TC{architectures need sizing for fast enough\\
       reaction to TIME-VARYING \emph{stimuli}, not just for a worst-case\\
       CONSTANT load} --- hence one has to know about dynamics.
 \item \vfill Enough on this for the moment, please experiment at home \& report.
 \end{itemize}
\end{frame}
